In this study, we adopt a deep learning approach for seismic data interpolation, employing three GANs: U-NET, LeNet-5, and Linear model. We assess their performance in image reconstruction using simplified MNIST and F-MNIST datasets. The generators and discriminators, built with U-NET, LeNet-5, or Linear model, iteratively learn features between masked datasets and the original dataset over 10 training epochs, generating reconstructed images at varying resemblance levels. We compare GANs' interpolation results through qualitative examination of output features and quantitative assessment using MSE losses on validation sets. U-NET consistently yields the most refined and accurate reconstruction, while LeNet-5 and Linear model capture the general shapes of handwritten digits and fashion items. Looking ahead to seismic data interpolation, we anticipate GANs to provide satisfactory results with more complex neural network architectures and fine-tuning of hyper-parameters. GANs offer flexibility by not requiring any velocity models or relying on linearity/sparsity assumptions. However, challenges such as training instability, feature loss due to missing information, and high-frequency details recovery need to be addressed.
\\\\
In future research, we plan to extend our GANs interpolation approach to seismic datasets. An exploratory avenue would involve experimenting with CycleGANs, potentially incorporating U-NET with additional layers for enhanced performance. Additionally, exploring different masking schemes, such as determining the maximum consecutive columns to be masked while retaining features, could provide valuable insights into improving overall performance.