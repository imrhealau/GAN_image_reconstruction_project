This study focuses on exploring the applicability of deep learning, particularly Generative Adversarial Networks (GANs), on the seismic trace interpolation problem. We first compare conventional seismic interpolation methods with the machine learning counterpart. Subsequently, we evaluate the performance of GANs in the context of image reconstruction, leveraging simplified datasets. Given the complexity and computational demands of seismic datasets, we opted to simulate the seismic problem using the more manageable 28x28 pixel MNIST and Fashion-MNIST datasets. These datasets were augmented by introducing masked columns, mimicking the missing trace scenario encountered in seismic data. We employ three GANs, namely Linear model, LeNeT-5, and U-NET, to reconstruct MNIST and Fashion-MNIST images. Our evaluation involves qualitative assessments and examination of losses. Notably, U-NET emerges as the most effective, producing high-definition results closest to the original images, as evidenced by MSE losses. Through this approach we gained insights into the effectiveness of GANs in handling image reconstruction and their potential in addressing seismic trace interpolation. However, challenges like training instability and feature loss due to missing information highlight the need for further model and hyper-parameters refinement in handling larger and more complex dataset.
\\\\